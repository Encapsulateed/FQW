\documentclass[a4paper,12pt]{article}
\usepackage[utf8]{inputenc}
\usepackage[T2A]{fontenc}
\usepackage[russian]{babel}
\usepackage{graphicx}
\usepackage{enumitem}
\usepackage{hyperref}
\usepackage{geometry}
\geometry{left=25mm,right=25mm,top=25mm,bottom=25mm}
\usepackage{titlesec}
\usepackage{longtable}
\usepackage{booktabs}

\title{Архитектурные решения (ADR) и функциональные требования (FR)}

\begin{document}

\maketitle
\tableofcontents
\newpage

\section{Архитектурные решения (ADR)}

\subsection{ADR-001: Выбор мобильной платформы и фреймворка}

\textbf{Контекст:}\\
Приложение предназначено для геологов, проводящих полевые исследования. Требуется мобильное приложение, способное работать в офлайн-режиме, обеспечивать высокую отзывчивость интерфейса и синхронизироваться с серверной частью.

\textbf{Варианты решения:}
\begin{itemize}[leftmargin=2cm]
    \item[--] \textbf{Нативная разработка (Android/iOS):} Максимальное использование возможностей платформ, однако требуется поддержка двух отдельных кодовых баз, что увеличивает временные затраты.
    \item[--] \textbf{Кроссплатформенные фреймворки (React Native, Xamarin):} Единая кодовая база снижает затраты, но могут возникнуть ограничения по производительности и доступу к некоторым нативным API.
    \item[--] \textbf{Flutter:} Обеспечивает высокую производительность благодаря собственному рендеринговому движку, позволяет создавать кроссплатформенные приложения с единой кодовой базой и имеет широкие возможности для кастомизации интерфейса, а также встроенную поддержку офлайн-режима.
\end{itemize}

\textbf{Решение:} Использовать \textbf{Flutter}.

\textbf{Обоснование:}
\begin{itemize}
    \item Экономия ресурсов за счёт единой кодовой базы.
    \item Высокая производительность и отзывчивость интерфейса.
    \item Гибкость в разработке и возможность быстрой итерации.
\end{itemize}

\textbf{Последствия:}
\begin{itemize}
    \item Быстрая разработка и выпуск обновлений.
    \item Возможность интеграции с нативными модулями при необходимости.
\end{itemize}

\subsection{ADR-002: Выбор серверной платформы}

\textbf{Контекст:}\\
Серверная часть отвечает за обработку запросов мобильного приложения, построение профилей местности по изолиниям, аутентификацию пользователей и интеграцию с внешними сервисами. Требуются высокая производительность, масштабируемость и безопасность.

\textbf{Варианты решения:}
\begin{itemize}[leftmargin=2cm]
    \item[--] \textbf{ASP.NET Core:} Высокая производительность, современные стандарты безопасности, богатый набор библиотек и средств для масштабирования, а также простая интеграция с внешними системами.
    \item[--] \textbf{Node.js (Express/Koa):} Хорош для быстрого прототипирования и эффективен при I/O-операциях, однако может быть менее эффективен при вычислительно затратных операциях и требует дополнительных мер безопасности.
    \item[--] \textbf{Java (Spring Boot):} Надёжная и масштабируемая платформа, широко используемая в корпоративных системах, но требует более сложной настройки и сопровождения.
\end{itemize}

\textbf{Решение:} Использовать \textbf{ASP.NET Core}.

\textbf{Обоснование:}
\begin{itemize}
    \item Оптимизирован для работы с большими нагрузками и масштабирования.
    \item Встроенные средства для обеспечения безопасности (аутентификация, авторизация).
    \item Лёгкость интеграции с внешними сервисами и базами данных.
\end{itemize}

\textbf{Последствия:}
\begin{itemize}
    \item Формирование единого REST API для мобильного и потенциальных веб-клиентов.
    \item Упрощение поддержки и модернизации серверной части.
    \item Возможность быстрого реагирования на изменяющиеся требования.
\end{itemize}

\subsection{ADR-003: Выбор базы данных}

\textbf{Контекст:}\\
Приложению необходимо хранить данные о рельефе, профили местности, геоданные и информацию о пользователях. Важен вопрос хранения профилей местности, включая возможность объектного представления данных.

\textbf{Варианты решения:}

\textbf{PostgreSQL с расширением PostGIS}
\begin{itemize}[leftmargin=1.5cm]
    \item \textbf{Структурированность:} Реляционная модель обеспечивает строгую схему и целостность данных.
    \item \textbf{Геопространственные возможности:} PostGIS предоставляет мощный инструментарий для сложных пространственных запросов и анализа (например, построение профилей по изолиниям).
    \item \textbf{Гибкость хранения:} Поддержка JSON/JSONB позволяет хранить объектные данные в сочетании с фиксированной схемой для критичных данных.
    \item \textbf{Надёжность:} Полная ACID-совместимость гарантирует корректность транзакций.
\end{itemize}

\textbf{MongoDB}
\begin{itemize}[leftmargin=1.5cm]
    \item \textbf{Гибкость схемы:} Документная база не требует фиксированной схемы, что удобно для хранения сложных, вложенных структур в формате JSON.
    \item \textbf{Объектное хранение:} Естественное представление данных в виде JSON-объектов.
    \item \textbf{Масштабируемость:} Хорошая поддержка горизонтального масштабирования.
    \item \textbf{Геопространственная поддержка:} Обеспечивает базовые геопространственные операции, но не обладает таким же уровнем аналитических возможностей, как PostGIS.
\end{itemize}

\textbf{Решение:} Использовать \textbf{PostgreSQL с расширением PostGIS}.

\textbf{Обоснование:}
\begin{itemize}
    \item Требуется мощный инструментарий для проведения сложных геопространственных запросов и анализа.
    \item Необходима высокая надёжность и целостность данных.
    \item Поддержка JSONB обеспечивает достаточную гибкость для хранения динамичных данных без перехода на полностью документную СУБД.
\end{itemize}

\textbf{Последствия:}
\begin{itemize}
    \item Максимальное использование геопространственных возможностей для построения профилей.
    \item Стабильное хранение и обработка критичных данных.
    \item Возможность адаптации структуры данных за счёт поддержки JSONB.
\end{itemize}

\subsection{ADR-004: Интеграция с внешними сервисами}

\textbf{Контекст:}\\
Для построения точных профилей местности и корректного отображения карт необходим доступ к внешним данным.

\textbf{Варианты решения для интеграций:}

\textbf{Геоданные API:}
\begin{itemize}[leftmargin=1.5cm]
    \item \textbf{OpenTopoData:} Предоставляет точные высотные данные для построения профилей.
    \item \textbf{Mapbox Terrain API:} Альтернативный сервис с высококачественными данными рельефа.
\end{itemize}

\textbf{Картографический API:}
\begin{itemize}[leftmargin=1.5cm]
    \item \textbf{Google Maps:} Широко распространённый сервис с обширной функциональностью и надежностью.
    \item \textbf{OpenStreetMap (OSM):} Бесплатная альтернатива, позволяющая гибко настраивать отображение карт.
\end{itemize}

\textbf{Решение:} Интегрировать \textbf{OpenTopoData} для геоданных и \textbf{Google Maps} (с возможностью использования OSM в зависимости от лицензионных условий) для картографических слоёв.

\textbf{Обоснование:}
\begin{itemize}
    \item Обеспечение актуальности и точности данных для построения профилей.
    \item Высокая стабильность и масштабируемость сервиса Google Maps.
    \item Гибкость выбора альтернативного решения (OSM) для снижения затрат при необходимости.
\end{itemize}

\textbf{Последствия:}
\begin{itemize}
    \item Повышение точности построения профилей за счёт качественных данных.
    \item Необходимость управления API-ключами и соблюдения лицензионных требований.
    \item Резервирование на случай недоступности основного сервиса.
\end{itemize}

\subsection{ADR-005: Пользовательская аутентификация и идентификация}

\textbf{Контекст:}\\
Для обеспечения безопасности, персонализации опыта и управления доступом к функциям приложения требуется надёжная система аутентификации пользователей.

\textbf{Варианты решения:}
\begin{itemize}[leftmargin=2cm]
    \item[--] \textbf{Identity Server (ASP.NET Core):} Интегрированное решение для реализации OAuth2/OpenID Connect, позволяющее обеспечить высокий уровень безопасности и гибкость в управлении доступом.
    \item[--] \textbf{Альтернативные решения (Auth0, Okta):} Облачные сервисы с богатым функционалом, однако сложны в разработке.
    \item[--] \textbf{Собственная разработка:} Разработка кастомного решения для аутентификации, что может увеличить время разработки и снизить общую безопасность.
\end{itemize}

\textbf{Решение:} Использовать \textbf{Identity Server}.

\textbf{Обоснование:}
\begin{itemize}
    \item Прямая интеграция с ASP.NET Core упрощает реализацию и поддержку.
    \item Поддержка современных стандартов безопасности (OAuth2/OpenID Connect).
    \item Нет необходимости в дополнительных затратах на сторонние облачные сервисы.
\end{itemize}

\textbf{Последствия:}
\begin{itemize}
    \item Обеспечение надёжной аутентификации и авторизации пользователей.
    \item Персонализация пользовательского опыта и управление доступом к функциям приложения.
    \item Возможность масштабирования системы управления доступом по мере роста нагрузки.
\end{itemize}

\newpage
\section{Функциональные требования (FR)}

Ниже приведён перечень функциональных требований, вытекающих из бизнес-задач проекта и архитектурных решений:

\subsection*{FR-001: Интерактивное построение профилей местности}
\begin{itemize}[leftmargin=1cm]
    \item Система должна принимать входные данные в виде геоданных и изолиний.
    \item Автоматически генерировать профиль местности на основе полученных данных.
    \item Обеспечить возможность ручного редактирования и корректировки профиля.
\end{itemize}

\subsection*{FR-002: Кроссплатформенность мобильного приложения}
\begin{itemize}[leftmargin=1cm]
    \item Мобильное приложение должно работать на платформах Android и iOS.
    \item Обеспечивать единообразный пользовательский интерфейс и функциональность на всех устройствах.
\end{itemize}

\subsection*{FR-003: Офлайн-режим и синхронизация данных}
\begin{itemize}[leftmargin=1cm]
    \item Приложение должно поддерживать офлайн-режим с локальным кэшированием данных.
    \item При восстановлении соединения обеспечить автоматическую синхронизацию с сервером.
\end{itemize}

\subsection*{FR-004: Безопасность и аутентификация пользователей}
\begin{itemize}[leftmargin=1cm]
    \item Реализовать надёжную систему аутентификации и авторизации через Identity Server с использованием стандартов OAuth2/OpenID Connect.
    \item Обеспечить защиту персональных данных и разграничение прав доступа.
\end{itemize}

\subsection*{FR-005: Интеграция с внешними сервисами}
\begin{itemize}[leftmargin=1cm]
    \item Система должна интегрироваться с геоданными API (например, OpenTopoData) для получения высотных данных.
    \item Обеспечить интеграцию с картографическими API (например, Google Maps или OSM) для отображения картографических слоёв.
    \item Управлять API-ключами и предусмотреть резервирование в случае недоступности основного сервиса.
\end{itemize}

\subsection*{FR-006: Экспорт данных и генерация отчетов}
\begin{itemize}[leftmargin=1cm]
    \item Обеспечить экспорт построенных профилей и данных в форматы PDF, CSV, DXF.
    \item Реализовать возможность генерации аналитических отчетов по местности.
\end{itemize}

\subsection*{FR-007: Мониторинг и логирование}
\begin{itemize}[leftmargin=1cm]
    \item Внедрить системы мониторинга, логирования и алертинга для отслеживания производительности и оперативного реагирования на ошибки.
\end{itemize}

\newpage 

\section{C4 Диаграмма}
\begin{figure}[!ht]
    \centering
    \includegraphics[width=0.9\linewidth]{arch-C4.png}
    \caption{C4 Диаграмма архитектуры приложения геологов.}
    \label{fig:c4diagram}
\end{figure}

\end{document}
